\documentclass{article}
\usepackage[utf8]{inputenc}
\usepackage[spanish]{babel}
\usepackage{listings}
\usepackage{graphicx}
\graphicspath{ {images/} }
\usepackage{cite}

\begin{document}

\begin{titlepage}
    \begin{center}
        \vspace*{1cm}
            
        \Huge
        \textbf{Taller - Nociones de la memoria del computador}
            
        \vspace{1.5cm}
        
        \textbf{Diego Andrés Zuluaga Alzate}
        
        \vspace{1.5cm}
        
        \begin{figure}[h]
        \includegraphics[width=6cm]{logoudea.png}
        \centering
        \label{fig:logoudea}
        \end{figure}
            
        \vspace{0.7cm}
            
        \Large
        Universidad de Antioquia\\
        Medellín\\
        Septiembre de 2020
            
    \end{center}
\end{titlepage}

\tableofcontents

\section{La memoria del computador}
¿que es la memoria del computador?, es una pregunta frecuente que nos hacemos en la cotidianidad, sabemos que esta allí para que el computador pueda funcionar pero no sabemos mucho más, la memoria es la encargada de almacenar información durante un periodo de tiempo, dispuesta allí porque se le dio una tarea a cumplir al microprocesador, la memoria guarda temporalmente esa información u orden que le demos al equipo y al acabar con el proceso  la información vuelve a su lugar original  y es borrada de la memoria para que no ocupe espacio algo que ya ha sido utilizado.\cite{memoria}
 
 \vspace{0.4cm}

\section{Los tipos de memoria} \label{contenido}

Esta sección es para ver qué pasa con los comandos 
que definen texto

El paquete también agrega un comportamiento especial 
a <<estas marcas para hacer citas textuales>> tal como 
lo indican las reglas de la RAE. \cite{dirac}

\begin{lstlisting}
#include <stdio.h>
#define N 10
/* Block
 * comment */

int main()
{
    int i;

    // Line comment.
    puts("Hello world!");
    
    for (i = 0; i < N; i++)
    {
        puts("LaTeX is also great for programmers!");
    }

    return 0;
}
\end{lstlisting}

A continuación se presenta el logo de C++ Figura (\ref{fig:cpplogo})

\begin{figure}[h]
\includegraphics[width=4cm]{cpplogo.png}
\centering
\caption{Logo de C++}
\label{fig:cpplogo}
\end{figure}

En la sección de teoremas (\ref{contenido})

\section{Conclusión} \label{conclulsion}

\bibliographystyle{IEEEtran}
\bibliography{references}

\end{document}
